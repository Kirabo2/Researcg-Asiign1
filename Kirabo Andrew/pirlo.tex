\documentclass[10pt]{article}
\usepackage{zed-csp,graphicx,color}%from



\begin{document}

  \begin{titlepage}
  \begin{figure}[h]
  \centerline{\small MAKERERE 
  \includegraphics[width=0.1\textwidth]{muk_log} UNIVERSITY}
\end{figure}
\centerline{COLLEGE OF COMPUTING AND INFORMATIC SCIENCES}
\paragraph{•}
\centerline{DEPARTMENT OF COMPUTER SCIENCE\\}
\paragraph{•}

\centerline{COURSEWORK: RESEARCH METHODOLOGY(BIT 2207)\\}
\paragraph{•}

\centerline{LECTURER: MR.ERNEST MWEBAZE}
\paragraph{•}

\centerline{TOPIC\\}EXAMINATION MALPRACTICES IN UNIVERSITIES. \\
\paragraph{•}
\centerline{COMPILED BY: \
 KIZITO ANDREW}
 \paragraph{•}
\centerline{STUDENT NUMBER : 216017100}
\paragraph{•}
\centerline{REGISTRATION NUMBER:16/U/6236/PS}
\paragraph{•}
\begin{flushright}
    Signature ....................\\
    DATE: $ FEBRUARY,8^{TH},2018$
\end{flushright}

  \end{titlepage}
  \pagenumbering{roman}
  \tableofcontents
  \newpage
  \pagenumbering{arabic}
  \section{INTRODUCTION}
    \subsection{BACKGROUND TO THE STUDY}
  Education in the general sense covers the whole life of an individual from birth till death or from cradle to grade that shows that education is as old as man on earth. The formal school system is greatly influenced in its result on the lives of all who pass through it.
  \subsection{ STATEMENT OF THE PROBLEM}
  The occurrence of examination malpractice at university level of educational stratum possess the greatest threat to the validity and reliability of any examination and consequently to the authenticity and recognition of certificate issued. The numerous examination malpractice in the University over the years has become a growing concern since cheating is such a longstanding and global problem inherent by human beings. Effort should therefore be directed towards controlling cheating behaviours and also finding the possible causes of the problem among COCIS students in Makerere University.
  \subsection{PURPOSE OF THE STUDY}
  The main purpose of this study is to find out the various forms of examinations malpractice among Universities like Makerere University(A case study of COCIS) and also to find out the causes in order to proffer the kind of counselling strategies for curbing the undesirable behaviour.
  \subsection{SIGNIFICANCE OF THE STUDY}
  The implementation of this finding will not only expose the extent to which students are involved in examination malpractice but will also suggest some remedies or a lasting solutions to this academic dishonesty.
This study is also expected to help university administrators, lecturers and guidance counsellors to curb or control cheating behaviour in the University.

  \subsection{RESEARCH QUESTIONS}
  The research questions for this study are stated below:
- Is there any difference between students who cheat and who did not cheat in an examination?
- Does examination malpractice contribute to educational development?
- Why do students indulge in examination malpractice?
- Are students aware of the penalty for examination malpractice?
- What roles have the university counsellors played so far in trying to curb these behaviours?
  \subsection{SCOPE/LIMITATION OF STUDY}
  
This study will concentrate on examination malpractice, their causes, effects and possible solution in  the selected college in the University

  \subsection{DEFINITION OF TERMS}
  In view of the fact that different meanings can be assigned will apply to the following words as used in this research work.
- EXAMINATION : A test of capacity and knowledge. It is a determinants of a learner’s strength and weakness necessary for his/her academic adjustment and work life.
- MALPRACTICE : It is a behaviour of a person contrary to laid down code of conduct.
- CHEATING :Cheating is refer to a type of examination malpractice, which takes place in the  personal problems.
  \subsection{LITERATURE REVIEW}
  Introduction This chapter will discuss the influence technology has had on interpersonal behaviour in students. Change in Communication Methods Before the Internet and the use of 
  \section{SUMMARY, DISCUSSION and RECOMENDATION}
  \subsection{SUMMARY}
  Research has indicated that students are spending more time communicating with each other through technology. This could mean that students will
be less likely to understand interactions that have subtleties beyond the words
themselves. Research has supported the notion that students are using technology to communicate more as technology progresses.

  \subsection{DISCUSSION}
  Technology will continue to advance, and given the recent changes in communication, face-to-face communication could continue to dwindle. If this is the
case, it may become increasingly difficult for people to pick up on the meaning
of body language and facial expression. 

  \subsection{RECOMMENDATION}
 More empirical research would benefit this social phenomenon. There does not
seem to be a lot of research measuring a decrease in ability to differentiate body
language and facial cues, nor is there much research on how this decrease would
impact social interactions. In particular, using social media to communicate.



\end{document}
















